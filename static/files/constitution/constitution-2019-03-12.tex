\documentclass{article}
\usepackage[a4paper, total={6in, 8in}]{geometry}
\usepackage{datetime}

\title{Constitution of The Archimedeans}
\date{\formatdate{12}{3}{2019}}
\begin{document}
\maketitle

\small\noindent{Adopted at the Constitutional General Meeting on \formatdate{4}{3}{1981},\\
amended at the Annual General Meeting on \formatdate{3}{3}{1982},\\
at the Extraordinary General Meeting on \formatdate{10}{6}{1984},\\
at the Annual General Meeting on \formatdate{6}{3}{1985},\\
at the Annual General Meeting on \formatdate{8}{3}{1989},\\
at the Annual General Meeting on \formatdate{7}{3}{1990},\\
at the Annual General Meeting on \formatdate{6}{3}{1991},\\
at the Extraordinary General Meeting on \formatdate{2}{5}{1993},\\
at the Annual General Meeting on \formatdate{2}{3}{1994},\\
at the Annual General Meeting on \formatdate{8}{3}{2000},\\
at the Annual General Meeting on \formatdate{7}{3}{2001},\\
at the Extraordinary General Meeting on \formatdate{29}{10}{2003},\\
at the Annual General Meeting on \formatdate{2}{3}{2004},\\
at the Extraordinary General Meeting on \formatdate{8}{5}{2005},\\
at the Extraordinary General Meeting on \formatdate{8}{2}{2006},\\
at the Annual General Meeting on \formatdate{6}{3}{2007},\\
and at the Annual General Meeting on \formatdate{12}{3}{2019}.
}

\section{Interpretation}

\begin{enumerate}
\item If any provision of the Constitution should be found to conflict with the
then current Statutes and Ordinances and Regulations and Proctorial
Edicts of the University then it shall be deemed void insofar as it does so
conflict.
\item There shall be Standing Orders of the Society. Any provision of the Standing Orders shall govern the conduct of Members as if it were a provision
of the Constitution, save that if it conflicts with any provision of the Constitution then it shall be void insofar as it does so conflict.
\item A copy of the Constitution and the Standing Orders shall be available on
the Society's website. A definitive copy shall be deposited with the Junior
Proctor.
\item In case of doubt, the Junior Proctor shall interpret the Constitution.
\end{enumerate}

\section{Name and Objects}

\begin{enumerate}
\item The Society shall be called ``The Archimedeans''.
\item The Objects of the Society shall be
  \begin{enumerate}
  \item to promote enjoyment and understanding of Mathematics among students of all disciplines, and to further the cause of Mathematics and
  Mathematicians in the University and elsewhere;
  \item to co-operate with and cause co-operation between the College Mathematical Societies;
  \item to co-operate with the Faculty of Mathematics in the University;
  \item to hold lectures and other events of interest to its Members;
  \item to publish two journals, which shall be called ``Eureka'' and ``QARCH'',
  and other publications of mathematical interest.
  \end{enumerate}
\item The activities of the Society shall have no unfair political, religious, racial
or sexual bias.
\end{enumerate}


\section{Membership}

\begin{enumerate}
\item The Society shall consist of Honorary Members, Ordinary Members and
Associate Members.
\item Persons may be invited to Honorary Membership by the Committee at
its discretion, or by the Society at a General Meeting. If the invitation is
accepted, such Membership shall last for five years from the first day of
October following the issue of the invitation.
\item Ordinary Members shall be those members of the University who hold a
currently valid Membership subscription as defined in section 4.
\item The Committee may, at its discretion, allow non-members of the University to become Associate Members of the Society, for which the subscription shall be identical to that for Ordinary Membership.
\item Membership shall cease in each of the following circumstances:
  \begin{enumerate}
  \item There ceases to be a category of Membership for which the person in
  question satisfies the requirements.
  \item The person in question resigns from the Society, by written notification to the Internal Secretary.
  \item A General Meeting of the Society (see section 10) dismisses the Member from Membership.
  \end{enumerate}
\item Members, and only Members, may attend any meetings of the entire Society\footnote{This excludes, for example, committee meetings.}. Members of the University Faculty of Mathematics may attend
without charge meetings of the entire Society at which no business is conducted. Other non-Members may be admitted to meetings of the entire
Society at which no business is conducted, at the discretion of the Internal Secretary\footnote{see Section \ref{committees}}, on payment of a sum to be decided by the Committee before each
meeting.
\end{enumerate}


\section{Subscriptions}

\begin{enumerate}
\item Subscriptions to Ordinary or Associate Membership shall be valid
\begin{description}
\item[either] until the seventh day of the second Full Michaelmas Term following
payment if payment is made between the beginning of the Easter
Term and the beginning of Michaelmas Full Term, or until the seventh
day of the first Full Michaelmas Term following payment otherwise
(``annual subscription'');
\item[or] without expiry (``life subscription'').
\end{description}
The payment required for each of the categories of subscription shall be
specified in the Standing Orders.
\item There shall be no subscription for Honorary Membership.
\end{enumerate}


\section{Committees}
\label{committees}

\begin{enumerate}
\item The affairs of the Society shall be governed by a Committee consisting
only of the following Officers, who must be Ordinary Members resident in
the University\footnote{during full term}, and of whom the first five must be distinct:
  \begin{enumerate}
  \item President,
  \item Vice-President,
  \item External Secretary,
  \item Junior Treasurer,
  \item Internal Secretary,
  \item Publicity Officer,
  \item Events Manager(s)\footnote{there can be up to two people occupying this role},
  \item Webmaster,
  \item Sponsorship Officer.
  \end{enumerate}
\item The President shall:
  \begin{enumerate}
  \item uphold the Objects of the Society;
  \item chair all meetings of the Society (with the possible exception of Extraordinary General Meetings, and with the likely exception of Meetings of the Publications Subcommittee (see below), which shall be
  chaired by the Business Manager), or appoint a deputy to chair the
  meeting if unavoidably absent from the same;
  \item be responsible for the entertainment of any guests of the Society;
  \item be responsible for the Annual Dinner of the Society;
  \item seek contact with the mathematical societies of other universities;
  \item ensure the execution of the duties of any vacant Committee post until
  it is filled;
  \item deal with the correspondence of the Society that does not fall clearly inside the remit of another officer;
  \item manage any property of the Society.
  \end{enumerate}
\item The Vice-President shall:
  \begin{enumerate}
  \item be responsible for the booking of rooms and the obtaining of equipment for Meetings;
  \item oversee the functioning of any Subcommittees of the Society (and
  thus have right of attendance at any Subcommittee Meetings);
  \item be accountable to the Committee for the functioning of the Subcommittees of the Society, and, in particular, for the (non-)publication
  of the Society Journals (see section 6).
  \item take on the responsibilities of the Business Manager (see section \ref{societyjournals}) should this role be unfilled;
  \item be responsible for the provision of refreshments at Meetings;
  \item execute the duties of the post of President should it fall vacant, until
  it is filled.
  \end{enumerate}
\item The External Secretary shall:
  \begin{enumerate}
  \item invite speakers to address Speaker Meetings;
  \item be responsible for the booking of overnight rooms for Speakers when
  necessary;
  \item be responsible for arranging Committee dinners after Speaker Meetings.
  \end{enumerate}
\item The Junior Treasurer shall:
  \begin{enumerate}
  \item answer to the Senior Treasurer for the financial affairs of the Society;
  \item keep true and accurate accounts of the Society;
  \item inspect on a regular basis on behalf of the Senior Treasurer all monies
  held in the name of the Society and managed by others (who shall
  be obliged to keep true and accurate accounts of such monies, and
  deliver them to the Junior Treasurer on request);
  \end{enumerate}
\item The Internal Secretary shall:
  \begin{enumerate}
  \item maintain Membership Lists and mailing lists;
  \item co-ordinate the recruiting of new Members;
  \item co-ordinate the Society's presence at a Freshers' Fair, should one take
  place in a Michaelmas Term;
  \item co-operate in the distribution of other Society publications;
  \item ensure that the Society complies with Data Protection legislation;
  \item inform the Junior Proctor of the University, the Senior Treasurer,
  and the Officers, of changes in the Constitution and of changes of
  Officers;
  \item be responsible for keeping minutes of all meetings except Subcommittee Meetings (with the minutes of Business Meetings kept in a
  different book from those of other Meetings), or appointing a deputy
  to keep minutes of a meeting if unavoidably absent from the same;
  \item maintain the Annals of the Society\footnote{The Annals of the Society
  	shall provide a complete and accurate account of the activities of the
  	Society, save for Speaker Meetings, which need only be minuted if there is a
  	Member willing to take minutes.}, and make them available to Members;
  \item be responsible for the Archives of the Society.
  \end{enumerate}
\item The Publicity Officer shall:
  \begin{enumerate}
  \item co-ordinate, and manage, the advertising of Meetings of the Society;
  \item co-operate with the Internal Secretary in recruiting new Members;
  \item maintain a social media presence for the Society.
  \end{enumerate}
\item The Events Manager(s) shall:
  \begin{enumerate}
  \item be responsible for events other than Speaker Meetings, except where
  otherwise defined;
  \item co-ordinate, and arrange, social events for Society members;
  \item co-ordinate, and arrange, where appropriate, social events with other
  relevant societies, within and outside the university.
  \end{enumerate}
\item The Webmaster shall:
	\begin{enumerate}
	\item maintain the website of the Society;
	\item keep an electronic record of the activities of the Society;
	\item co-operate with the Internal Secretary to ensure the printing of a membership card for every Member.
	\end{enumerate}
\item The Sponsorship Officer shall:
	\begin{enumerate}
		\item be responsible for finding and maintaining sponsorship for the Society;
		\item oversee the organization of Sponsors;
		\item ensuring compliance with the sponsorship agreements in place.
	\end{enumerate}
\item The Committee may, having created such a post in the Standing Orders,
appoint a willing Member to execute any of the duties for which a Committee Member is only defined to be responsible above. Members occupying
such posts shall be known as Agents.
\item The Committee shall meet upon the request of any Officer. A Subcommittee shall meet upon the request of any of its Members, or upon the
request of the Vice-President.
\item Each Member of the Committee or a Subcommittee shall be given at
least 24 hours notice of its Meetings, normally by the Vice-President,
and a provisional agenda shall be made available at least 24 hours in
advance of the meeting by the person responsible for taking minutes, unless
all Members of the Committee or Subcommittee agree to meet with less
notice.
\item At a meeting of the Committee or a Subcommittee, each Member who
holds at least one post in the relevant Committee or Subcommitee shall
have exactly one vote. Decisions shall be made by simple majority, subject
to the condition that three votes must be cast in favour of the decision at
a Committee Meeting, and two at a Subcommittee Meeting.
\item The Plenum shall contain:
  \begin{enumerate}
  \item all Officers of the Society;
  \item all Subcommittee Members;
  \item all Agents;
  \item the Senior Treasurer;
  \item any other persons, at the invitation of the Committee.
  \end{enumerate}
\item A Committee Member may invite any Member of the Plenum to a Committee Meeting. The Vice-President shall normally invite the entire Plenum
to the following Special Committee Meetings:
  \begin{enumerate}
  \item during the Lent Full Term, after the Annual General Meeting;
  \item during the first thirteen days of Easter Full Term, or on the day
  preceding them;
  \item during the first thirteen days of Michaelmas Full Term, or during the
  three days preceding them;
  \item after the division of Lent Term, before the Annual General Meeting.
  If the entire Plenum has been invited, with at least 24 hours notice, to a
  Committee Meeting, then during that meeting a Plenum Meeting may be
  held. Plenum Meetings may not be convened in any other way. The only
  function of a Plenum Meeting shall be to consider and make changes to
  the Standing Orders.
  \end{enumerate}
\item Simultaneous meetings of the Committee and/or Subcommittees may not
take place, unless all those entitled to attend the multiple simultaneous
meetings (whether by right or by invitation) consent.
\end{enumerate}


\section{The society journals}
\label{societyjournals}

\begin{enumerate}
\item There shall exist a permanent Publications Subcommittee of the Society.
It shall consist of a Business Manager, a Subscriptions Manager, the Editors of Eureka and QARCH, the Eureka Assistant Editor and the Eureka
Online Editor. Of these, the Business Manager shall be distinct from the
Eureka and QARCH Editors.
\item The Business Manager shall:
  \begin{enumerate}
  \item chair all meetings of the Publications Subcommittee, or appoint a
  deputy to chair the meeting if unavoidably absent from the same;
  \item deal with all correspondence relating to Eureka and QARCH that
  does not fall clearly inside the remit of another Member of the Publications Subcommittee;
  \item co-ordinate and be responsible for the production and printing of
  Eureka and QARCH;
  \item with the assistance of the Committee, distribute copies of Eureka and
  QARCH to those who have requested them other than by means of
  a subscription account held with the Society;
  \item make all reasonable efforts to attract advertising and sponsorship for
  Eureka and QARCH;
  \item manage the finances of Eureka and QARCH (in such a way that they
  comply with the relevant requirements in the Junior Treasurer's job
  description);
  \item be answerable to the Committee for the other Members of the Publications Subcommittee.
  \end{enumerate}
	Should this post fall vacant, it is the responsibility of the Vice-President (see section \ref{committees}) to fulfil these duties.\\
\item The Subscriptions Manager shall:
  \begin{enumerate}
  \item manage subscription accounts for Eureka and QARCH held with the
  Society;
  \item deal with requests from subscribers for back issues of the Society
  journals;
  \item on publication of an issue of Eureka or QARCH, take charge of its
  distribution to subscribers (including those entitled to free copies in
  the case of Eureka) with the assistance of the Committee.
  \end{enumerate}
	Should this position fall vacant, it is the responsibility of the Internal Secretary (see section \ref{committees}) to fulfil these duties.\\
\item The Editors of Eureka and QARCH shall:
  \begin{enumerate}
  \item collect articles for their respective journals;
  \item have overall editorial control over their respective journals;
  \item be answerable to the Business Manager for the (non-)publication of
  their respective journals;
  \item have authority to grant licences for the reproduction of works in
  which copyright or database right is held by the Archimedeans, in
  accordance with item 8 of this section.
  \end{enumerate}
\item The Eureka Assistant Editor shall assist the Eureka Editor in the performance of their duties, and shall usually take minutes of Publications
Subcommittee Meetings.
\item The Eureka Online Editor shall endeavour to make past issues of Eureka
available electronically, under the supervision of the Committee.
\item An Honorary Member shall be entitled to receive free of charge all copies
of Eureka published during their Honorary Membership.
\item The Editors of Eureka and QARCH shall have authority to grant licences
to perform specified acts restricted by copyright or database right relating
to works, including changes to works and typographical arrangements, in
which the Society owns or may acquire any copyright or database right (including prospective future copyright or database right on works or changes
to works or typographical arrangements yet to be created), provided that:
  \begin{enumerate}
  \item The work has been accepted for publication in the journal under the
  editor's control;
  \item The licence is granted to the original author of the work (or, in the
  case of works with multiple authors, to any of the original authors of
  the work) only;
  \item The licence grant is non-exclusive\footnote{It may be sub-licensable; particularly it may be sub-licensable to parties to whom the
  Editor is not permitted by this section to grant licences.};
  \item The licence grant takes place not more than one year after the publication of the issue of the journal in which the work is included,
  and not more than five years after the acceptance of the work for
  publication;
  \item The licence grant only permits reproduction in accordance with the
  Objects of the Society.
  \end{enumerate}
\item The Committee shall have authority to grant licences to perform acts
restricted by copyright relating to works in which the Society holds copy
right, and to grant licences to perform acts restricted by database right
relating to works in which the Society holds database right, only restricted
by the provision that such grants must further the Objects of the Society.
\end{enumerate}



\section{Election and tenure of officers and appointment of Subcommittees}

\begin{enumerate}
\item Unless otherwise terminated, the tenures of Officers shall last from the
end of the Annual General Meeting (see next section) in which they are
elected (or the time at which they are co-opted, if applicable) until the
end of the following Annual General Meeting.
\item The tenures of Subcommittee Members and Agents shall last until the
end of the second Lent Special Committee Meeting, unless terminated
beforehand.
\item Subcommittee Members and Agents shall be appointed by the Committee.
\item Any Member of the Society shall be eligible to hold a Subcommittee post.
\item There shall be an election of Officers at the Annual General Meeting (see
next section). These Officers shall assume responsibilities immediately
after the closure of the Annual General Meeting.
\item Nominees for election as Officers shall be Ordinary Members resident in
the University. Nominations shall be made in writing to any officer at least twenty-four hours in advance of the commencement of the Annual General Meeting.
\item Nominations may be withdrawn at any time before the election takes
place, but only by the candidate.
\item Twenty four hours before the commencement of the Annual General Meeting, the Society shall publish a list of the valid nominations received. In
the event that there are no valid nominations for an Office within twenty-four hours of the election, the requirement that nominations be submitted
in advance shall be waived for that Office.
\item Those eligible for election to an Office may stand for election to more than
one Office, and retiring Officers who remain Ordinary Members resident in
the University may stand for re-election. However, if at the time of election
for an Office there is a candidate whose election would contravene any
provision of this Constitution, then that candidate shall be disqualified.
\item In the event of more than one nomination for election to any Office, there
shall be a secret ballot to be conducted in accordance with the University's
Single Transferable Vote regulations as altered in the appendix to this
Constitution. Only Ordinary Members resident in the University shall be
eligible to vote in elections.
\item Elections of Officers and appointments of Subcommittee Members and
Agents shall occur in the reverse order in which the Offices and posts are first
mentioned in section\footnote{for posts that are mentioned in Section \ref{committees}; appointments to such posts should be prior to other appointments}, and shall take place after the results of the previous election or appointment have been declared.
\item Candidates for each election or appointment shall have left the room during the voting for that election or appointment.
\item If an Office is not filled during a General Meeting and falls vacant at
the end of it, or if an Office falls vacant at any other time, then the
remaining Officers shall meet within seven days of Full Term to co-opt
a willing Ordinary Member resident in the University to that Office. If
a Subcommittee post falls vacant at any time other than at the end of
the second Lent Special Committee Meeting, then the Officers shall meet
within seven days of Full Term to appoint a willing Member to that post.
\item In addition to the above, Officers shall cease to serve on the Committee
under each of the following circumstances:
  \begin{enumerate}
  \item the Officer resigns in writing to the President (or, in the case of the
  President, to the Internal Secretary);
  \item the Officer ceases to be an Ordinary Member resident in the University;
  \item the Officer is dismissed from Office at an Extraordinary General
  Meeting (see next section);
  \item the Officer is requested to resign by all the other Officers at a Committee Meeting, having been given notice of the intention to serve
  this request at least 72 hours before the meeting.
  \end{enumerate}
\item In addition to the above, Subcommittee Members and Agents shall cease
to serve on a Subcommittee or as an Agent under each of the following
circumstances:
  \begin{enumerate}
  \item the Subcommittee Member or Agent resigns in writing to the Vice-President;
  \item the Subcommittee Member or Agent ceases to be a Member of the
  Society;
  \item the Subcommittee Member or Agent is relieved of their post by the
  Committee;
  \item the Subcommittee Member or Agent is dismissed from their post by
  an Extraordinary General Meeting (see next section).
  \end{enumerate}
\end{enumerate}


\section{General meetings}

\begin{enumerate}
\item Only Ordinary Members resident in the University shall be eligible to vote
at General Meetings.
\item The Annual General Meeting shall usually be held in the afternoon of
the penultimate Wednesday of the Lent Full Term within the University
precincts. If necessary, the Committee may alter this date by up to five
days (earlier or later). The date, time and venue of the Annual General
Meeting shall have been advertised on the website or by email at least
fourteen days in advance.
\item The business of the Annual General Meeting shall be conducted in the
following order:
  \begin{enumerate}
  \item reading, amendment if necessary, and acceptance of all minutes not
  already accepted of any General Meeting held prior to the day of the
  Annual General Meeting;
  \item presentation and acceptance of annual reports, to include:
    \begin{enumerate}
    \item a general report by the President on the functioning of the Society over the previous year;
    \item an interim report for the current financial year by the Junior
    Treasurer, to include Society accounts and a provisional budget for the next
    financial year;
    \item an audited set of accounts for the previous financial year, inconsolidated Society accounts, and a comparison of the
    budget to actual expenditure;
    \item a report by the Business Manager on the Society journals;
    \end{enumerate}
  \item any other business specifically designated to be conducted at a General Meeting by this Constitution;
  \item election of Officers for the coming year;
  \item any other business.
  \end{enumerate}
\item An Extraordinary General Meeting shall be held between fourteen and
twenty-eight days of Full Term after the Committee shall have passed a
resolution to hold such a meeting, or the Internal Secretary shall have received a
petition to hold such a meeting from at least ten Members.
It shall be held within the University precincts, at a time and place calculated to be to the convenience of Ordinary Members resident in the
University, and shall be advertised at least fourteen days in advance.
\item At the beginning of an Extraordinary General Meeting, the Members
present shall appoint somebody to chair the meeting from those Members present and willing to chair the meeting. The person chairing the
meeting shall then appoint a willing Member to take minutes.
\item An Extraordinary General Meeting may conduct any Society Business,
including that listed below.
\item The following business may be conducted by General Meetings, and may
not be conducted in any forum other than as specified in this Constitution:
  \begin{enumerate}
  \item dismissal from Membership of any Member;
  \item reinstatement to Membership of any person previously dismissed
  from Membership;
  \item election of any person to Honorary Membership;
  \item dismissal from Office of any Officer;
  \item dismissal from a post of any Subcommittee Member;
  \item amendment of the Constitution or Standing Orders.
  \end{enumerate}
Notice of such business must have been given seven days in advance of the
meeting, unless motions are amendments of previous motions, or deemed
called-for by the chair as a result of the dismissal of previous motions. It
may only be transacted with a quorum of ten Members eligible to vote, and shall require
a majority of at least two thirds of those present and voting.
\item Unless otherwise specified, voting on matters of dispute at General Meetings shall be by a simple majority of those present and voting. The person
chairing the meeting shall have a casting vote only in the event of a tie,
and shall not have a vote otherwise.
\end{enumerate}


\section{Indemnity}

\begin{enumerate}
\item The Society shall have a Senior Treasurer who shall be ex officio an Honorary Member. Should the post of Senior Treasurer fall vacant, within
fourteen days of Full Term the Committee shall invite a willing Member
of the University Faculty of Mathematics to accept the post.
\item The Senior Treasurer shall audit the Society's accounts, and shall be responsible for the financial affairs of the Society as required by the Statutes
and Ordinances and Regulations and Proctorial Edicts of the University.
\item No Member other than the Senior Treasurer shall be held responsible for
debts incurred on behalf of the Society. No debt shall be incurred on behalf
of the Society without the approval of at least four Officers including the
Junior Treasurer and with the written consent of the Senior Treasurer.
The Society shall not be held responsible for debts otherwise incurred.
\item Members of the Society may be reimbursed for reasonable expenses incurred on behalf of the Society, at the discretion of the Committee. No
Officer, Subcommittee Member or Agent shall be paid any fee for any
service.
\item The Society shall be responsible in all circumstances for Eureka and QARCH. The back numbers of all Society publications, the collection of magazines and books owned by
the Society, and any monies in the
cash reserves or bank accounts of these concerns, shall be the property of
the Society.
\item Such controversial opinions as may be expressed in any publication of the
Society shall not be deemed to be those of the Committee.
\end{enumerate}


\section{Amendment}

\begin{enumerate}
\item The Constitution may be amended only at a General Meeting. The notice
to be given to all Voting Members may just outline the nature of the
proposed changes provided that they shall have been published in full for
at least the seven days of Full Term preceding the General Meeting.
\item The Standing Orders may be amended at a Plenum Meeting. Amendment shall be by simple majority of those present and voting; each person
present who is a member of the Plenum shall have exactly one vote.
\end{enumerate}


\appendix
\section{Single transferable vote procedure for the election of officers}

(Adapted from the University Statutes and Ordinances, 2016)
\begin{enumerate}
\item Every elector in giving their vote:
  \begin{enumerate}
  \item must enter on the voting paper (see diagram), against the figure 1,
  the name of the candidate to whom they give first preference;
  \item may in addition enter on the voting-paper, against the figures 2, 3,
  and so on, the names of any other candidates in order of preference.
  \end{enumerate}

\begin{tabular}{|l|p{10cm}|}
\hline
Order of preference  & Candidate \\ \hline
1                    &           \\ \hline
2                    &           \\ \hline
3                    &           \\ \hline
4                    &           \\ \hline
5                    &           \\ \hline
6                    &           \\ \hline
7                    &           \\ \hline
8                    &           \\ \hline
\end{tabular}


\item A voting-paper shall not be valid unless the elector's first preference is
legibly and unambiguously expressed.
\item There shall be made available the following instructions:
Enter against the figure 1 the name of the candidate to whom you give
first preference.
You may also enter, against the figures 2, 3, and so on, the names of other
candidates in the order of your preference for them, continuing until you
are indifferent. The order of your preferences is crucial. A later preference
can be considered only if an earlier preference has been excluded because
of insufficient support.
\item \textit{Counting of votes: the first stage}
  \begin{enumerate}
  \item The voting-papers shall be sorted into parcels according to the first
  preferences recorded for each candidate, any invalid papers being set
  aside.
  \item The number of first preference votes for each candidate and the total
  number of valid votes (i.e. voting-papers) shall be determined.
  \item The returning officer shall then determine the number of votes sufficient to secure the election of a candidate (the ‘quota’), by dividing
  the total number of valid votes by 2, the result being rounded up to
  the next whole number above, if it is not an exact whole number.
  \end{enumerate}
\item \textit{Election of a candidate}

If at any stage of the count a candidate is credited with a number of votes
equal to or exceeding the quota they shall be deemed to be elected,
except that, if the number of candidates attaining the quota is greater
than the number of places to be filled, the two candidates who attained
the quota at the most recent stage of the count shall be deemed not to
be elected, and the returning officer shall proceed to the next stage of the
count.
\item \textit{Counting the votes: subsequent stages}

Subsequent stages of the count shall be conducted as follows. The candidate with the smallest number of votes shall be excluded from the poll,
and their votes shall be transferred to the continuing candidates next
in order of the voters preference, in accordance with the provisions of Regulations 7 and 8. Each transfer shall be deemed to constitute a further
stage of the count.
\item \textit{The exclusion of a candidate}

If a vacancy is not filled as a result of the first preferences recorded, the
candidate or candidates credited with the smallest number or numbers of
votes shall be excluded from the poll, as follows:
  \begin{enumerate}
  \item The two or more candidates credited with the smallest number of
  votes shall be excluded together if the total number of votes of such
  two or more candidates does not exceed the number of votes credited
  to the candidate with the next smallest number of votes.
  \item Otherwise, the candidate credited with the smallest number of votes
  shall be excluded if the number of votes of such a candidate does not
  exceed the number of votes credited to the candidate with the next
  smallest number of votes.
  \item If the two or more candidates credited with the smallest number of
  votes have each the same number of votes, the candidate who had the
  smallest number at the earliest stage at which they had an unequal
  number shall be excluded. If such two or more candidates have been
  credited with the same number of votes at all stages of the count, the
  returning officer shall determine by lot which candidate to exclude.
  \end{enumerate}
\item The exclusion of a candidate or candidates from the poll shall be effected
in the following manner:
  \begin{enumerate}
  \item The voting-papers of the candidate or candidates to be excluded shall
  be sorted into parcels according to the next available preferences
  for continuing candidates, any papers on which no next available
  preference is expressed being set aside.
  \item The returning officer shall determine the number of the papers in
  each parcel, and the number of the non-transferable papers.
  \item Each continuing candidate shall be credited with the number of
  any papers transferred to them, and the number of any non-transferable papers shall be added to the previous non-transferable
  total.
  \item After the transfer of the parcels of papers, the returning officer shall
  ascertain which candidates (if any) are deemed to be elected in accordance with the provisions of Regulation 5.
  \end{enumerate}
\item \textit{The final stages}
  \begin{enumerate}
  \item When the proposed exclusion of a candidate or candidates would
  reduce the number of continuing candidates to one, the continuing
  candidate shall be deemed to be elected.
  \item When the vacancy can be filled under this regulation, no further
  transfers of votes shall be made, and the remaining continuing candidate or candidates shall be formally excluded from the poll.
  \end{enumerate}
\item In publishing the result of the election the returning officer shall include
a notification of any transfer of votes made under these regulations, and
of the total number of votes credited to each candidate after any such
transfer.
\item Any candidate or any candidate’s representative may, at any time during
the counting of the votes, either before the commencement or after the
completion of any transfer of votes, request the returning officer to re-examine and recount the papers of any or all candidates (not being papers
set aside at a previous transfer as finally dealt with), and the returning
officer shall forthwith re-examine and recount the same accordingly; the
returning officer shall also have discretion to recount votes either once or
more often in any case in which they are not satisfied as to the accuracy
of any previous count; provided that nothing contained in this regulation
shall make it obligatory for the returning officer to recount the same votes
more than once.
\item If any question shall arise in relation to any transfer of votes, the decision
of the returning officer, whether expressed or implied by their acts,
shall be final.
\item For the purposes of these regulations an Ordinary Member who is not
standing in any election, appointed by the person chairing the meeting at
which the elections are held, shall be the returning officer.
\item In these regulations:
  \begin{enumerate}
  \item `valid voting-paper' means a voting-paper on which a first or an only
  preference is legibly and unambiguously expressed;
  \item `invalid voting-paper' means a voting-paper on which no first preference is expressed, or on which any first preference is void for uncertainty;
  \item `continuing candidate' means a candidate not yet elected and not
  excluded from the poll;
  \item `next available preference' means the next preference in order, passing
  over any earlier preferences for candidates who have already been
  excluded;
  \item `transferable paper' means a voting-paper on which a next available
  preference for a continuing candidate is legibly and unambiguously
  expressed;
  \item `non-transferable paper' means a voting-paper on which no next available preference for a continuing candidate is expressed, or on which
  any next available preference is void for uncertainty.
  \end{enumerate}
\end{enumerate}

\end{document}
